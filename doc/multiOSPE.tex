%published in PoC||GTFO Issue 0x01

\documentclass{article}
\usepackage{listings}
\usepackage{hyperref}
\begin{document}
\lstset{language={[x86masm]Assembler}}
\section{making a multi-Windows PE}
\subsection{evolution of the PE loader}
The {\em Portable Executable} loader of Windows has slowly evolved and became progressively more restrictive. Many oddities that worked in the past are killed by the next version: for example, the notorious TinyPE\footnote{\href{http://www.phreedom.org/research/tinype/}{Creating the smallest possible PE executable}} cannot work after Windows XP, as next revisions of Windows enforce that the {\tt OptionalHeader} is not truncated in the file, thus forcing a TinyPE to be padded (shame!) until 252 bytes (or 268 bytes in 64 bits systems), so that it still works. Windows 8 also brought new requirements like $AddressOfEntryPoint <= SizeOfHeaders $ when $ AddressOfEntryPoint \neq 0$, so any old-school packer such as FSG\footnote{Fast Small Good, by bart/xt} will not work anymore.

So, there are many real-life examples of Windows binaries that will just stop working with the next version. It is, on the other hand, much harder to create a Windows binary that would still work, but differently --- from a loader perspective, not just a version detection in the code. This will imply that Windows is not a single evolving OS, but rather successive distinguishable yet related OSes. While I already\footnote{differentiating OS versions by putting  \href{https://corkami.googlecode.com/svn-history/r889/trunk/asm/PE/tls_aoiOSDET.asm}{TLS AddressOfIndex in an Imports descriptor}.} did something similar, it was only able to differentiate between XP and next generations of Windows.

\subsection{a look at PE relocations}
PE relocations have been known to produce some weirdness: some Mips-related type have been present on all platforms, x86/Sparc/Alpha/Mips included, one type appeared and disappeared in Windows 2000. Typically, PE relocations are limited to a simple role: whenever a binary needs to be relocated, the standard type 3 (mapped to {\tt HIGH\_LOW}\footnote{relocation names actually use the {\tt IMAGE\_REL\_BASED\_} prefix}) relocations are applied by adding the delta $LoadedImageBase - HeaderImageBase$ to each 32 bits immediate.

However, more types are actually usable, and 2 of them present some interesting differences between OS values that we can use~:
\begin{description}
	\item[type 9] This one has a very odd formula\footnote{see Roy G Biv's \href{http://spth.virii.lu/v3/vessel/display/articles/roy g biv/vcode2.txt}{Virtual Code}}, where it can modify 16 bytes altogether under Windows 7, while it only modifies 32 bits under XP. Sadly, it's not supported anymore under Windows 8. It's mapped to {\tt MIPS\_JMPADDR16}, {\tt IA64\_IMM64} and {\tt MACHINE\_SPECIFIC\_9}.
	\item[type 4] This type is the only one that takes a parameter, which is ignored under versions older than Windows 8. It's mapped to {\tt HIGH\_ADJ}.
	\item[type 10] This type is supported by all versions of Windows, but it will still help us. It's mapped to {\tt DIR64}.
\end{description}

So, one relocation type (type 9) is behaving differently under XP and 7, but has no effect under older version than Windows 8.
On the other hand, type 4 behaves differently under Windows 8, unlike previous versions. In this case, we can use it to turn an unsupported type 9 into a supported type 10. This is possible because relocations are applied directly in memory, where they can freely modify the following relocation entries.

\subsection{Implementation}
So, here's our plan:
\begin{enumerate}
\item give a user-mode PE a kernel-mode ImageBase, to force relocations
\item add the standard relocations (for code)
\item make a relocation of type 4 be applied to a further type 9 relocation entry
\begin{itemize}
\item under XP or W7, the type 9 relocation will keep its type, with an offset of $0f00h$
\item under W8, the type will be changed to a supported 10, and the offset will be changed to $0000h$
\end{itemize}
\item we end up with a memory location, that is either:
\begin{description}
\item[XP] modified on 32b ($00004000h$)
\item[W7] modified on 64b ($08004000h$)
\item[W8] left unmodifed ($00000000h$), because a completely different location was modified by a relocation type 10.

\end{description}
\end{enumerate}

\begin{lstlisting}
;relocation type 4, to patch unsupported relocation type 9 (Windows 8)
block_start1:
    .VirtualAddress dd relocbase - IMAGEBASE
    .SizeOfBlock dd BASE_RELOC_SIZE_OF_BLOCK1

    ; offset +1 to modify the Type, parameter set to -1
    dw (IMAGE_REL_BASED_HIGHADJ  << 12) | (reloc4 + 1 - relocbase), -1 
BASE_RELOC_SIZE_OF_BLOCK1 equ  - block_start1
\end{lstlisting}

\begin{lstlisting}
; our type9/type10 relocation block: 
; type 10 under Windows8, 
; type 9 under XP/W7, where it behaves differently
block_start2:
    .VirtualAddress dd relocbase - IMAGEBASE
    .SizeOfBlock dd BASE_RELOC_SIZE_OF_BLOCK2

; 9d00h will turn into 9f00h or a000h
reloc4 dw (IMAGE_REL_BASED_MIPS_JMPADDR16 << 12) | 0d00h
BASE_RELOC_SIZE_OF_BLOCK2 equ $ - block_start2
\end{lstlisting}

We now have a memory location modified transparently by the loader, with a different value depending on the OS version. It could be extended to generate different code, this is left as an exercise for the reader.

\subsubsection{Building instructions}
The entire binary is defined by hand, thus a single invokation of Yasm\footnote{\href{http://yasm.tortall.net/}{The Yasm Modular Assembler Project}} is required:
$ yasm\ -o\  relocOSdet.exe\ relocOSdet.asm$ 
is required. The final SHA-1 should be $65427add61db59bdb65daf28c72cc14b2ae309aa$.
\end{document}